%********** Chapter 6 **********
\chapter{Camera Transformation Calculation}
\section{Introduction}
In the previous chapter, we explained how to extract feature points from a frame then find matches in the next frame. Now, we explain how we calculate a transformation between the two sets of feature points, this transformation is then applied on the second point cloud to align it with the global point cloud.

Tranformation calculation is a well known mathematical problem, where we calcluate a transformation matrix that when applied to the first set of points will result in the second set of points.

Some of the methods used for transformation calculation are iterative and others are closed form solutions, we have found that applying a closed form solution then using the results as a seed for the iterative solution gives better and faster results than applying only one method.

\section{Horn's method}

We Consider that problem of transforming the second point cloud to the first point cloud identical to the problem of transforming one coordinate system to another, this problem is named the Absolute Orientation Problem, we used the closed form solution proposed by Horn.

The transformation between the two coordinate systems is assumed to be rotation and translation, we denote it as $F_{AB} = [R | t]$

\begin{figure}[htb]
\centering
\includegraphics{Transformation_Calc/transformation_AB.png}
\caption{One coordinate system transformed to the other}
\label{fig:transformation_AB}
\end{figure}

\subsection{Quaternions}
Quaternions are a mathematical object of the form
$$ Q = s + ix + jy + kz = s + v $$
where s, x, y, z $\in$ R, and i, j, k are mutually orthogonal imaginary units with the
following composition rule

\begin{table}[htb]
\centering
\begin{tabular}{ l | l  l  l }
  & i  & j  & k \\ \hline
  i& -1& k & -j\\
  j& -k& -1& i\\
  k& j & -i& -1\\
\end{tabular}
\caption{i,j and k composition rule}
\label{tab:ijkcomposition}
\end{table}

We now define several basic operations on quaternions:
\begin{itemize}
\item Addition and Substraction:
     Given two quaternions $q_1 = (s_1, v_1)$ and $q_2 = (s2, v2)$ their addition/subtraction is defined as
$$q = q_1 + q_2 = (s_1 + s_2, v_1 + v_2)$$
\item Multiplication:
Given two quaternions $q_1 = (s_1, v_1)$ and $q_2 = (s_2, v_2)$ their multiplication is defined as
$$q_1 * q_2 = (s_1 + v_1) * (s_2 + v_2) = s_1s_2 + s_1v_2 + s_2v_1 + v_1 * v_2$$

Then compute:
$$v_1 * v_2 = (iv_{1x} + jv_{1y} + kv_{1z}) * (iv_{2x} + jv_{2y} + kv_{2z})$$
$$ = (-v_{1x}v_{2x} - v_{1y}v_{2y} - v_{1z}v_{2z}) +$$
$$ i(v_{1y}v_{2z} - v_{1z}v_{2y}) + $$
$$ j(v_{1z}v_{2x} - v_{1x}v_{2z}) + $$
$$ k(v_{1x}v_{2y} - v_{1y}v_{2x}) $$
$$ = -(v_1 \cdot v_2) + (v_1 \times v_2) $$

Then
$$ q = q_1 * q_2 = [(s_1s_2 � v_1 \cdot v_2);(s_1v_2 + s_2v_1 + v_1 \times v_2)]$$

\end{itemize}

We have just defined quaternions and their basic operations, let's explain how they work as rotation operators

Let's first explain how rotations are expressed using Euler angles then we talk about the problem of gimbal lock and how quaternions avoid it.

Euler angles represent a 3D rotation as a rotation around the 3D axes $x,y $ and $z$, so a rotation is expressed as:

$$ R_{AB} = R_z(\omega_z)R_y(\omega_y)R_x(\omega_x)$$
$$ = 
\begin{bmatrix}
   \cos(\omega_z) & -\sin(\omega_z) & 0 \\
   \sin(\omega_z) & \cos(\omega_z) & 0 \\
   0 & 0 & 1 \\
\end{bmatrix}
\begin{bmatrix}
   \cos(\omega_y) & 0 & \sin(\omega_y) \\
   0 & 1 & 0 \\
   \sin(\omega_y) & 0 & \cos(\omega_y) \\
\end{bmatrix}
\begin{bmatrix}
	1 & 0 & 0 \\
   0 & \cos(\omega_x) & \sin(\omega_x) \\
   0 & \sin(\omega_x) & \cos(\omega_x)  \\
\end{bmatrix}
$$

\normalsize
$$ = 
\begin{bmatrix}
   \cos(\omega_z)\cos(\omega_y) & \cos(\omega_z)\sin(\omega_y)\sin(\omega_x) - \sin(\omega_z)\cos(\omega_x) & \cos(\omega_z)\sin(\omega_y)\cos(\omega_x) + \sin(\omega_z)\sin(\omega_x)  \\
   \sin(\omega_z)\cos(\omega_y) & \sin(\omega_z)\sin(\omega_y)\sin(\omega_x) + \cos(\omega_z)\cos(\omega_x)  &  \sin(\omega_z)\sin(\omega_y)\cos(\omega_x) - \sin(\omega_z)\sin(\omega_x)  \\
   -\sin(\omega_y) &  \cos(\omega_y)\sin(\omega_x)  & \cos(\omega_y)\cos(\omega_x) \\

\end{bmatrix}
$$

\large
\pagebreak
if we have a rotation matrix and want to get the rotation angles around the axes we use the following relations:

$$\omega_y = \mathrm{atan2}(-r_{31}, \sqrt{r_{11}^2 + r_{21}^2}) $$
$$\omega_z = \mathrm{atan2}(r_{21}/\cos{\omega_y}, r_{11}/\cos{\omega_y}) $$
$$\omega_x = \mathrm{atan2}(r_{32}/\cos{\omega_y}, r_{33}/\cos{\omega_y}) $$


the problem of gimbal lock happens when a rotation causes two of the axes are in the same plane, for instance, if $\omega_y = \pi/2$ then we will not be able to extract the angles in the previous relations.


\begin{figure}[htb]
\centering
\includegraphics[scale=0.45,keepaspectratio=true]{Transformation_calc/gimbal_lock.png}
\caption{Gimbal lock happens when two rotation axes are in the same plane}
\label{fig:gimbal_lock}
\end{figure}

this happens because Euler angles defines a 3D rotation as 3 rotations around the three axes, which causes a loss of degree of freedoms when two axes are in the same plane, this doesn't happen when we use quaternions because they define rotation around a unit vector not around the three axes $x, y$ and $z$

\pagebreak

The conjugate norm and inverse are given by the following three equations:

$$\bar{q} = [s, -v] $$
$$ N(q) = \| q \| = \sqrt{q * \bar{q}} = \sqrt {s^2 + |v|^2} $$
$$ q^{-1} = \frac{1}{q} = \frac{1}{q} * \frac{\bar{q}}{\bar{q}} = \frac{\bar{q}}{N(q)^2}$$

if $ N(q) = 1 $ then the quaternion is referred to as a unit quaternion and $\bar{q} = q^{-1}$

\subsection{Quaternions as rotational operators}

Given two vectors $a$ and $b$ and a unit quaternion of the form $q = [\cos \frac{\theta}{2}, n \sin \frac{\theta}{2}] $ the general rotation of $a$ into $b$ about an arbitrary unit axis $n$ by $\theta$ radians is given by the equation:

$$ b = q * a * q^{-1} $$
It can be proved that the previous equation is equivalent

\subsection{Closed Form Registration}

Given two sets of corresponding points in two different coordinate systems, we want to compute the transformation between the coordinate systems.
This solution assumes we have at least three correspondences between the two coordinate systems.

For any two corresponding points $p_l$ and $p_r$, the error in the tranformation can be defined as:
$$ e_i = p_r - R(p_l) - t $$

We want to minimize the sum of square errors over all points:
$$\sum_{i=1}^{n} \| e_i \| ^ 2$$

We will refer all measurements to the centroids given by:
$$ \mu_l = \frac{1}{n} \sum_{i=1}^{n} p_{l,i} $$
$$ \mu_r = \frac{1}{n} \sum_{i=1}^{n} p_{r,i} $$

The new coordinates are now:
$$ p'_{l,i} = p_{l,i} - \mu_l $$
$$ p'_{r,i} = p_{r,i} - \mu_r $$

the error term using the new coordinates becomes
$$ e_i = p'_r - R(p'_l) - t' $$
where
$$ t' = t - \mu_r + R\mu_l$$

The sum of square errors becomes
$$\sum_{i=1}^{n} \| e_i \| ^ 2 = \sum_{i=1}^{n} \|  p'_r - R(p'_l) - t' \| ^ 2$$
$$ = \sum_{i=1}^{n} \|  p'_r - R(p'_l) \| ^ 2 - 2t' \cdot \sum_{i=1}^{n}[p'_{r,i} - Rp'_{l,i}] + n\|t'\|^2 $$

Because $\sum_{i=1}^{n} p'_{r,i} = \sum_{i=1}^{n} p'_{l,i} = 0$ we notice that the middle term in the previous equation equals zero, we minimize the last term by making it equal to zero then
$$ t' = 0 $$
$$ \downarrow $$
$$ t = \mu_r - R\mu_l $$

We have now to minimize the first term
$$ \sum_{i=1}^{n} \|  p'_r - R(p'_l) \| ^ 2 = \sum_{i=1}^{n} \| p'_{r,i} \| ^ 2 - 2 \sum_{i=1}^{n} p'_{r,i} \cdot Rp'_{l,i} + \sum_{i=1}^{n} \| Rp'_{l,i} \| ^ 2$$

The first and third terms of the previous equation are constants independent of R because rotations preserve the vector norm. We minimize the error by maximizing the second term.

This can be expressed in the following figure:
\begin{figure}[htb]
\centering
\includegraphics{Transformation_calc/gemoetric_int.png}
\caption{Maixmizing the sum $ \sum{i=1}^{n} p'_{r,i} \cdot Rp'_{l,i} $ is equivalent to maximizing $ \sum{i=1}^{n} | p'_{r,i} | |Rp'_{l,i}| \cos\theta $. This sum is maximized when $\cos\theta = 1$, $\theta=0$. Gemoetrically we compute the rotation which minimizes the angle between the two vectors}
\label{fig:gemoetric_int}
\end{figure}

We express the equation using unit quaternions:
$$ \sum{i=1}^{n} (q * p'_{l,i} * \bar{q}) \cdot p'_{r,i} $$
$$ = \sum{i=1}^{n} (q * p'_{l,i}) \cdot (p'_{r,i} *q) $$

We use the matrix representation:

$$p'_{r,i} * q = \begin{bmatrix}
   0 & -x'_{r,i} & -y'_{r,i}& -z'_{r,i} \\
   x'_{r,i} & 0 & -z'_{r,i} & y'_{r,i} \\
   y'_{r,i}& z'_{r,i} & 0 & -x'_{r,i}  \\
   z'_{r,i} & -y'_{r,i} & x'_{r,i} & 0  \\
\end{bmatrix}q = \Re_{r,i}q $$

and

$$q * p'_{l,i} = \begin{bmatrix}
   0 & -x'_{l,i} & -y'_{l,i}& -z'_{l,i} \\
   x'_{l,i} & 0 & z'_{l,i} & -y'_{l,i} \\
   y'_{l,i}& -z'_{l,i} & 0 & x'_{l,i}  \\
   z'_{l,i} & y'_{l,i} & -x'_{l,i} & 0  \\
\end{bmatrix}q = \bar{\Re_{l,i}}q $$

So we have:

$$ \sum{i=1}^{n} (\bar{\Re_{l,i}}q) \cdot (\Re_{r,i}q) $$
$$ \Downarrow $$
$$ \sum{i=1}^{n} q^T\bar{\Re_{l,i}}^T \Re_{r,i}q $$
$$ \Downarrow $$
$$ q^T (\sum{i=1}^{n} \bar{\Re_{l,i}}^T \Re_{r,i})q $$
$$ \Downarrow $$
$$ q^T (\sum{i=1}^{n} N_i) q $$
$$ \Downarrow $$
$$ q^T N q$$

The matrix $N$ is symmetric as it is a sum of symmetric matrices. The vector $q$ which maximizes $q^TNq$ is the eigenvector corresponding to the most positive eigenvalue of the matrix $N$.

We construct the matrix N by:
$$ M = \sum{i=1}^{n} p'_{l,i}p{'T}_{r,i}$$
$$ = \sum{i=1}^{n} [p_{l,i}p_{r,i}^T] - \mu_l\mu_r^T$$
$$ = \begin{bmatrix}
   S_{xx} & S_{xy} & S_{xz} \\
   S_{yx} & S_{yy} & S_{yz} \\
   S_{zx} & S_{zy} & S_{zz}  \\
\end{bmatrix}$$
where
$$ S_{xx} \sum{i=1}^{n} x'_{l,i}x'_{r,i} $$
$$ S_{xy} \sum{i=1}^{n} x'_{l,i}y'_{r,i} $$
and so on. The matrix contains all the information needed to construct the matrix N
$$ N =  \begin{bmatrix}
   trace(M) & \Delta^T \\
   \Delta & M + M^t - trace(M)I_3 \\
\end{bmatrix}$$ 
where
$$ N =  \begin{bmatrix}
   (M - M^T)_{23} \\
   (M - M^T)_{31} \\
	(M - M^T)_{12} \\
\end{bmatrix}$$

\subsection{Results of Horn's method}

We have successfully implemented the previously explained method, and have reached results that (measure time of horn)
We noticed that this method works well when:
\begin{itemize}
\item The two frames are rich in features, meaning they have a high (specific) number of matches.
\item The features must also be well distributed over the image space, meaning that features are not only focused in one place in the two images.
\end{itemize}

if one or both of these conditions fail to exist, the results will not be correct and what happens is that the transformation calculated will be biased towards the feature points used in the calculations, this means that these features will be aligned well but the rest of the scene may or may not align in the right place.

This problem occurs a lot when the scanned scene contains a large continous space of the same surface like walls or floors, these surfaces don't contribute in the feature tracking thus are not considered in the transformation calculations, this is explained in the following images:


\begin{figure}[htb]
\centering
\includegraphics[scale=0.5, width =150mm,keepaspectratio=true]{Transformation_calc/matches_chairs.png}
\caption{We notice that most features are concentrated in the parts that contains chairs while the wall on the right has no features matched}
\label{fig:matches_chairs}
\end{figure}

this causes the transformation to be baised towards the chairs while neglecting the alignment of the walls

\begin{figure}[htb]
\centering
\includegraphics[scale=0.5, width =150mm,keepaspectratio=true]{Transformation_calc/chairs_horn.png}
\caption{Areas marked with yellow show how alignment fails in featureless areas}
\label{fig:chairs_horn}
\end{figure}
